\documentclass[10pt]{article}

\usepackage[tagged]{accessibility}

%\usepackage[cm]{fullpage}
\usepackage[lmargin=0.65in,rmargin=0.65in,
            tmargin=0.5in,bmargin=0.75in]{geometry}

\usepackage[parfill]{parskip}

\usepackage{hyperref}

\usepackage{palatino}

% URLs (special font for monospace)
\usepackage[defaultsans]{cantarell}
\usepackage[T1]{fontenc}


\newenvironment{itemsquish}
  { \begin{itemize}
    % set spacing between items
    \addtolength{\itemsep}{-0.25\baselineskip}
    % set spacing between lines
    \addtolength{\baselineskip}{-0.25\baselineskip} }
  { \end{itemize} }

\hypersetup{
  pdftitle={PHY 546: Python for Scientific Computing (Spring 2025)},
  pdflang={en-US}
}

\begin{document}

\begin{center}
{\LARGE \sffamily \bfseries PHY 546: Python for Scientific Computing} \\[1mm]
{\sffamily Spring 2025} \\[3mm]
{\em Instructor}\/: Michael Zingale, ESS 452, michael.zingale@stonybrook.edu \\
{\em Date/Location}\/: Mondays, 2:00--2:55~pm in Humanities 3018 \\
{\em Web:}\/ {\small \url{https://sbu-python-class.github.io/python-science}}
\end{center}

\subsection*{Learning Goals:}

The learning goal of this course is to enhance scholarship in an area
of active research, by learning how to apply python to problems
in your field.


\subsection*{Format:}

The class will be interactive throughout, with students working
through Jupyter notebooks together with the instructor during the
class.  Students will be expected to read and work through some basic
notebooks on their own before class, and then, in class, we will work
on projects together, sharing what we learn.

This class is {\em heavily} dependent on out-of-class discussion,
managed via slack.  All students are expected to participate in this
online discussion.

\subsection*{Credit:}

This is a 1-credit course.


\subsection*{Contacting the Instructor:}

{\em e-mail:} michael.zingale@stonybrook.edu ({add ``PHY 546'' to the
 start of the subject line of any e-mail}).
%
%% \begin{tabbing}
%% \noindent {\em office hours:} \= Tues.~~  \=1:00~pm \=to \=2:30~pm \\
%%                               \> Thurs.\ \>1:00~pm \>to \>2:30~pm
%% \end{tabbing}
%% Generally, I am in my office every day this semester, except for Wednesdays.



\subsection*{Texts:}

There are no required textbooks for this class.  Online information
will be linked to from the course webpage.

A nice text for you to read along with the class, if desired, is
{\em Effective Computation in Physics}\/ by Scopatz \& Huff.



\subsection*{Lecture Topics:}

We will (try to) discuss the following topics in the course:
%
\begin{itemsquish}
\item {\em Introduction to python} (4 lectures) \\ data structures and
  control statements, functions, classes, popular modules, Jupyter
  notebooks

\item {\em Software engineering practices} (1 lecture) \\
  including {\tt git} and github, and unit testing (with {\tt pytest})

\item {\em Introduction to the NumPy array library} (2 lecture)

\item {\em matplotlib / bokeh / plot.ly for visualization} (1 lecture)

\item {\em SciPy and numerical methods} (2 lectures)

\item {\em Introduction to SymPy} (1 lecture)

\item {\em Machine learning} (1 lecture)

\item {\em Building applications / packaging} (1 lecture)
\end{itemsquish}

\noindent Time-permitting, we will also discuss:
\begin{itemsquish}
\item {\em NetworkX}

\item {\em Pandas and the data frame} (1 lecture)


\item {\em Cython / Numba / extensions}

\end{itemsquish}

\noindent The actual course topics and time spent on each topic will depend on the
interest and the participation level of the class.


\subsection*{Computers:}
%
As we all use different systems for our research, we are not meeting
in a computer lab.  Instead, you should bring your own laptop to
class.  Information on how to install python on Windows, Mac OSX, and
Linux will be posted on the class webpage.

\noindent
If you don't have access to a computer that can run python, then you will be
able to follow along for most of the course content using a cloud platform.


\subsection*{Slack:}
%
We will use the slack team communication tool for all discussion.  You
will be added to the slack team at the start of the semester and then
can join in on the conversation at: {\sf phy546-S25}


\subsection*{Evaluation:}

Students are expected to attend the class and to contribute
to the slack discussions (by asking questions, proposing examples, or
providing demonstrations of their own).  As we meet only one hour per
week, students show plan on spending time outside of class reviewing
and practicing the material we discussed.

\noindent {\em \bfseries The primary place for participation is the slack team chat}.  This is the place to interact with
me and your classmates---ask anything, share examples, etc.

\noindent Letter grades will be based on the online participation.  A
rough guide is presented below:
\begin{itemize}
\item {\sf A\phantom{+}}: 10 (meaningful) postings to slack {\bf plus}
  a short code example to our class git repo showing how you can apply what
  we've discussed in class to your field.

\item {\sf A$-$}: 10 (meaningful) postings to slack

\item {\sf B$+$}: 5 postings to the slack

\item {\sf B\phantom{+}}:  3 postings to the slack
\end{itemize}
A post does not mean a ``me to''-type post, but something either
demonstrating a problem you don't understand (giving code), asking for
some detail from the lecture to be explained, sharing a neat trick you
found, answering a classmate's question, etc.  {\em \bfseries Note that asking
questions about the content counts just as much as providing
answers}---the idea is to have a discussion outside of class on the
material.



\subsection*{Student Accessibility Support Center Statement}

If you have a physical, psychological, medical, or learning disability
that may impact your course work, please contact the Student
Accessibility Support Center, Stony Brook Union Suite 107, (631)
632-6748, or at sasc@stonybrook.edu. They will determine with you what
accommodations are necessary and appropriate. All information and
documentation is confidential.

Students who require assistance during emergency evacuation are
encouraged to discuss their needs with their professors and the
Student Accessibility Support Center. For procedures and information
go to the following website:\\
{\small \url{https://ehs.stonybrook.edu//programs/fire-safety/emergency-evacuation/evacuation-guide-disabilities}}\linebreak
and search Fire Safety and Evacuation and Disabilities.


\subsection*{Academic Integrity}

Each student must pursue his or her academic goals honestly and be
personally accountable for all submitted work. Representing another
person's work as your own is always wrong. Faculty is required to
report any suspected instances of academic dishonesty to the Academic
Judiciary. Faculty in the Health Sciences Center (School of Health
Professions, Nursing, Social Welfare, Dental Medicine) and School of
Medicine are required to follow their school-specific procedures. For
more comprehensive information on academic integrity, including
categories of academic dishonesty please refer to the academic
judiciary website at\\
{\small \url{http://www.stonybrook.edu/commcms/academic\_integrity/}}


\subsection*{Critical Incident Management}

Stony Brook University expects students to respect the rights,
privileges, and property of other people. Faculty are required to
report to the Office of Student Conduct and Community Standards any
disruptive behavior that interrupts their ability to teach,
compromises the safety of the learning environment, or inhibits
students' ability to learn. Faculty in the HSC Schools and the School
of Medicine are required to follow their school-specific
procedures. Further information about most academic matters can be
found in the Undergraduate Bulletin, the Undergraduate Class Schedule,
and the Faculty-Employee Handbook.


\subsection*{Electronic Communication}

Email to your University email account is an important way
of communicating with you for this course.  For most students the
email address is `{\tt firstname.lastname@stonybrook.edu}'.
%, and the account can be accessed here.
{\em It is your responsibility to read your email received at this
  account.}  For instructions about how to verify your University
email address see this: \\
{\small \url{http://it.stonybrook.edu/help/kb/checking-or-changing-your-mail-forwarding-address-in-the-epo}}\\
If you choose to forward your University email to another account, we
are not responsible for undeliverable messages.


\subsection*{Religious Observances}

See the policy statement regarding religious holidays at\hfill\\
{\small \url{https://www.stonybrook.edu/commcms/registrar/calendars/religious_holidays.php#2022}} \linebreak
%

Students are expected to notify the course professors by email of
their intention to take time out for religious observance.  This
should be done as soon as possible but definitely before the end of
the `add/drop' period.  At that time they can discuss with the
instructor(s) how they will be able to make up the work covered.
\end{document}
